\documentclass[11pt]{article}
\usepackage{latexsym}
\usepackage{amsmath,amssymb,amsthm}
\usepackage{epsfig}
\usepackage[right=0.8in, top=1in, bottom=1.2in, left=0.8in]{geometry}
\usepackage{setspace}
\spacing{1.06}

\newcommand{\handout}[6]{
  \noindent
  \begin{center}
  \framebox{
    \vbox{\vspace{0.25cm}
      \hbox to 5.78in { {CS 240:\hspace{0.12cm} Reasoning Under Uncertainty (Fall 21)} \hfill #2 }
      \vspace{0.48cm}
      \hbox to 5.78in { {\Large \hfill #5  \hfill} }
      \vspace{0.42cm}
      \hbox to 5.78in { {#3 \hfill #4} }\vspace{0.25cm}
    }
  }
  \end{center}
  \vspace*{4mm}
}

\newcommand{\lecture}[5]{\handout{#1}{#2}{#3}{SI Worksheet:\hspace{0.08cm}#4}{Lecture #1}}

\newtheorem{theorem}{Theorem}
\newtheorem{corollary}[theorem]{Corollary}
\newtheorem{lemma}[theorem]{Lemma}
\newtheorem{observation}[theorem]{Observation}
\newtheorem{example}[theorem]{Example}
\newtheorem{definition}[theorem]{Definition}
\newtheorem{claim}[theorem]{Claim}
\newtheorem{fact}[theorem]{Fact}
\newtheorem{assumption}[theorem]{Assumption}

\newcommand{\E}{{\mathbb E}}
\DeclareMathOperator{\var}{Var}

\begin{document} 

\lecture{11-17}{Oct 18-Nov 10}{Instructor:\hspace{0.08cm}\emph{Profs Peter J. Hass and Jie Xiong}}{\emph{Juelin Liu}}


\section{Introduction}
The second mid-term will cover materials from lecture 11 to 17. This worksheet will summarize them. 

\section{Continuous Random Variables}
Continuous random variables can take a continuous range of values as outcomes.
We can use continuous random variables to approximate discrete random variables.
Using powerful calculus tools, we can significantly speed up the computation through approximation. (ex. Poisson vs Binominal) 

\subsection{Probability Density Function}
For a continuous random variable $X$, its probability density function $f_X(x)$ describes its probability at any real interval.
Perhaps the most striking property of PDF is that at any particular point, we have:
\begin{equation}
  P(X=k) = \int_k^k f_X(x) dx = F_X(k) - F_X(k) = 0
\end{equation}

Another property of PDF is:
$$P(-\infty < X < \infty) = \int_{-\infty}^{\infty} f_X(x) dx = 1$$

\subsection{Cumulative Distribution Function}
$F_X$ mentioned in equation (1) is the cumulative distribution function of $X$. 
It computes the probability that X is less than or equal to a given value: $P(X \leq k) = F_X(k)$. 

The probability of $X$ in a real interval can be considered as the $area$ under the PDF graph within that interval. 
$F_X(k)$ represents the area within the interval $(-\infty, k]$.
It can be computed through integration:
$$F_X(k) = \int_{-\infty}^{k} f_X(x) dx$$

Note that one use case of CDF is to analyze the maxima of independent random variables:
\begin{align*}
  P(max(X_1, X_2) \leq k) &= P(X_1 \leq k \: and \: X_2 \leq k) 
  \\ &= P(X_1 \leq k) P(X_2 \leq k)
  \\ &= F_{X_1}(k)F_{X_2}(k)
\end{align*}

\subsection{Expected Value \& Variance}
For a continuous random variable $X$, its expected value can be computed as the following:
$$E(X) = \int_{-\infty}^{\infty} xf_X(x) dx$$
\noindent Its variance can be computed as the following:
$$Var(X) = E((x - E(X))^2) = \int_{-\infty}^{\infty} (x - E(X))^2 f_X(x) dx$$
Notice that $E(X)$ is a constant, we can always move it to the outside of the integration like other constants:
\begin{align*}
  \int_{-\infty}^{\infty} (x - E(X))^2 f_X(x) dx &= \int_{-\infty}^{\infty} (x^2 - 2xE(X) + E(X)^2) f_X(x) dx 
  \\ &= \int_{-\infty}^{\infty} x^2 f_X(x) dx - 2E(X) \int_{-\infty}^{\infty} xf_X(x)dx + E(X)^2 \int_{-\infty}^{\infty} f_X(x)dx
  \\ &= E(X^2) - 2E(X) E(X) + E(X)^2 \times 1
  \\ &= E(X^2) - E(X)^2
\end{align*}
which means:
$$Var(X) = E(X^2) - E(X)^2$$

\section{Practice Problems}
\begin{enumerate}
  \item Let X be a continuous random variable with probability density function given by:
  \begin{equation*}
    f_X(x) = 
    \begin{cases}
      3x^2 & 0 \leq x \leq 1 \\
      0 & otherwise
    \end{cases}
  \end{equation*}
  Find $F_X$ and draw the graph for both $f_X$ and $F_X$. Is $f_X$ a valid probability density function?
  \item Given that 
  \begin{equation*}
    f_X(x) = \begin{cases}
      cx^2 & 0 \leq x \leq 2 \\
      0 & otherwise
    \end{cases}
  \end{equation*}
   a) Find the value of c for which f(x) is a valid density function. \\
   b) After determining the value of c, find $P(1 \leq x \leq 2)$ and $P(1 < x < 2)$. \\
   c) Find $E(X)$ and $Var(X)$. 
  \item The proportion of time per day that all checkout counters in a supermarket are busy is a random variable X with density function
  \begin{equation*}
    f_X(x) = 
    \begin{cases}
      cx^2(1-x)^4 & 0 \leq x \leq 1 \\
      0 & elsewhere
    \end{cases}
  \end{equation*}
  a) Find the value of c that makes $f_X(x)$ a probability density function. \\
  b) Find $E(X)$ and $Var(X)$. (hint: you can use your calculator)
  \item If $X$ is a continuous random variable such that $E((X - k)^2) < \infty$ for all k, show that $E((X - k)^2)$ is minimized when $k = E(X)$.
\end{enumerate}
\end{document}

