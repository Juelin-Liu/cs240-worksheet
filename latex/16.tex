\documentclass[11pt]{article}
\usepackage{latexsym}
\usepackage{amsmath,amssymb,amsthm}
\usepackage{epsfig}
\usepackage[right=0.8in, top=1in, bottom=1.2in, left=0.8in]{geometry}
\usepackage{setspace}
\newcommand{\notimplies}{%
  \mathrel{{\ooalign{\hidewidth$\not\phantom{=}$\hidewidth\cr$\implies$}}}}
\spacing{1.06}

\newcommand{\handout}[6]{
  \noindent
  \begin{center}
  \framebox{
    \vbox{\vspace{0.25cm}
      \hbox to 5.78in { {CS 240:\hspace{0.12cm} Reasoning Under Uncertainty (Fall 21)} \hfill #2 }
      \vspace{0.48cm}
      \hbox to 5.78in { {\Large \hfill #5  \hfill} }
      \vspace{0.42cm}
      \hbox to 5.78in { {#3 \hfill #4} }\vspace{0.25cm}
    }
  }
  \end{center}
  \vspace*{4mm}
}

\newcommand{\lecture}[5]{\handout{#1}{#2}{#3}{SI Worksheet:\hspace{0.08cm}#4}{Lecture #1}}

\newtheorem{theorem}{Theorem}
\newtheorem{corollary}[theorem]{Corollary}
\newtheorem{lemma}[theorem]{Lemma}
\newtheorem{observation}[theorem]{Observation}
\newtheorem{example}[theorem]{Example}
\newtheorem{definition}[theorem]{Definition}
\newtheorem{claim}[theorem]{Claim}
\newtheorem{fact}[theorem]{Fact}
\newtheorem{assumption}[theorem]{Assumption}

\newcommand{\E}{{\mathbb E}}
\DeclareMathOperator{\var}{Var}

\begin{document} 

\lecture{17}{Nov 8}{Instructor:\hspace{0.08cm}\emph{Profs Peter J. Hass and Jie Xiong}}{\emph{Juelin Liu}}


\section{Introduction}
In this lecture, we introduced the Game Theory.

\section{Game Theory}
Game theory studies what happens when self-interested players interact.
A player's benefits can be represented using a payoff matrix that maps the actions to real numbers.
% For example, let player $P_1$ has n different startegies and player $P_2$ has m different strategies. 
% There are $n \times m$ different combinations of strategies, and each gives some benefits to $P_1$ and $P_2$.
% We can use two matrices to represent the benefits for $P_1$ and $P_2$ respectively, or we can use a payoff matrix to combine the two.
\subsection{Elements In A Game}
There are several elements in a game
\begin{itemize}
  \item We limit the number of players to 2, so we have player A and player B.
  \item Each player has a set of strategies. 
  We denote the strategy chosen by player $A$ as $a$.
  \item A \textbf{strategy profile} $s = (a, b)$ specifies the strategy that is chosen by each player. 
  \item Each player has a payoff function (denoted as $u_A$ and $u_B$.) 
  Given the strategy profile, $u_A(s)$ returns the payoff for $A$, and $u_B(s)$ returns the payoff for $B$.
\end{itemize}
We assume each player knows the payoff function throughout the game including the payoff functions for the other players.

\subsection{Terminologies}
\textbf{Payoff Function}

\textbf{Dominant Strategies}
\begin{enumerate}
  \item Strategy $a$ \textbf{strictly dominates} $a^{\prime}$: \\
  Choosing $a$ always gives a better outcome than choosing $a^{\prime}$, no matter what the other players do.
  \item Strategy $a$ is \textbf{strictly dominant}: \\
  Strategy $a$ strictly dominates any other strategies of player $A$.
  \item Strategy $a^{\prime}$ is \textbf{strictly dominated} by strategy $a$: \\
  Strategy $a$ strictly dominates $a^{\prime}$.
\end{enumerate}
\textbf{Nash equilibrium}
is a stable state of the system that involves several interacting participants in which no player can gain by a change of strategy as long as all the other participants remain unchanged.
\begin{theorem}
Every game where each player has a finite number of strategies has at least one Nash equilibrium.
\end{theorem}
\noindent A strategy profile is a Nash equilibrium if $A$'s strategy $a$ is a 'best response' to what $B$ does and vice versa.

\subsection{Prisoner Dilemma}
In this setting, it's easy to see what will happen. 
Whatever A does, B is better off confessing. Whatever B does, A is better off confessing.
So they will both confess. This is because both players in the game have a dominant strategy (confess).

\subsection{Hawks and Doves}
In this setting, Neither A nor B has a dominant strategy. 
Even though there is no equilibrium in dominant strategies, it turns out that there exist at least two Nash equilibria: (H, D) and (D, H).
This shows there might exist multiple Nash equilibria in the game.

\subsection{Mixed Strategy}
Intuitively, a Nash equilibrium is established when all the players stick to a strategy profile because each player cannot get a better payoff by changing his/her strategy if other players stay the same.

The intuition of the mixed strategy is each player randomizes his/her strategy to nullify the other player's efforts to maximize their expected payoff.
For example, player A randomizes his/her strategy such that player B's expected payoff is irrelevant to how B chooses his/her strategy.
Player B can randomize his/her strategy such that player A's expected payoff is irrelevant to how A chooses his/her strategy.
Under this scenario, both A and B do not have incentive to change their strategies and this is a Nash equilibrium by definition.
% Of course, making other players obtain less payoff does not necessarily make you better off (unless the game is zero-sum).
% However, Nash equilibrium is not about finding the optimal strategies that maximize one player's payoff by definition.

\section{Problem}
1. Find the Nash equilibrium of a game with the following payoff matrix:
\begin{center}
  \begin{tabular}{|c|c|c|c|}
    \hline
    A, B  &  i    & j    & k     \\ \hline
    i     &  6, 6 & 10,15& 3, 7  \\ \hline
    j     &  12,3 & 5,  5& 2, 8  \\ \hline
    k     &  8, 0 & 20, 3& 4, 4  \\ \hline
  \end{tabular}
\end{center}

\noindent 2. Suppose A and B play tennis. 
If A serves to a side where B stands he loses the point, 
if A serves to a side where B does not stand, he wins the point.
The payoff matrix for this game is as follows:
\begin{center}
  \begin{tabular}{|c|c|c|}
    \hline
    A, B  &  Left & Right \\ \hline
    Left  &  -1,1 & 1, -1 \\ \hline
    Right &  1,-1 & -1, 1 \\ \hline
  \end{tabular}
\end{center}
\noindent Find a Nash equilibrium for this setting?

\noindent 3. For the Hawk and Dove example discussed in class, the following payoff matrix is given. 
If A and B play Hawk with probability p = q = 0.2 is a mixed-strategy Nash equilibrium, what is the value of a?
\begin{center}
  \begin{tabular}{|c|c|c|}
    \hline
    A, B  &  Left    & Right \\ \hline
    Left  &  -25,-25 & 50, 0 \\ \hline
    Right &  0,  50  & a, a \\ \hline
  \end{tabular}
\end{center}

\section{Answer}
1. Nash equilibrium at <k, k>
\begin{enumerate}
  \item For player A: k dominates i
  \item For player B: k dominates i and j
  \item For player A: k dominates j
\end{enumerate}

\noindent 2. Let's assume A randomize his strategy so he plays left with probability t and right with probability 1 - t and player B's payoff is irrelevant to his/her strategy. Then
$$t + (1-t)(-1) = (-1)t + (1-t)$$
$$t = 0.5$$
We can obtain B's mixed strategy in the same way.
So both players play each strategy with a probability of 0.5.

\noindent 3. 
$$-25q + 50(1-q) = 0 * q + a(1-q)$$
$$a = \frac{175}{4}$$
\end{document}

